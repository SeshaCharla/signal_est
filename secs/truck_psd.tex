\section{Power Spectral Density of Individual Truck Signals}

\begin{multicols}{2}
       \begin{figure}[H]
        \centering
        \includegraphics[width=0.48\textwidth]{./figs/bfr_smth/truck_psd/y1.png}
        \caption{PSD of $[NO_x]^{out}$}
       \end{figure}

       \begin{figure}[H]
        \centering
        \includegraphics[width=0.48\textwidth]{./figs/bfr_smth/truck_psd/u1.png}
        \caption{PSD of $[NO_x]^{in}$}
       \end{figure}
\end{multicols}

\begin{multicols}{2}
       \begin{figure}[H]
        \centering
        \includegraphics[width=0.48\textwidth]{./figs/bfr_smth/truck_psd/T.png}
        \caption{PSD of temperature}
       \end{figure}

       \begin{figure}[H]
        \centering
        \includegraphics[width=0.48\textwidth]{./figs/bfr_smth/truck_psd/F.png}
        \caption{PSD of mass flow rate}
       \end{figure}
\end{multicols}

\begin{figure}[H]
        \centering
        \includegraphics[width=0.6\textwidth]{./figs/bfr_smth/truck_psd/u2.png}
        \caption{PSD of $[NO_x]^{out}$}
\end{figure}


% ============================================================================

\subsection{Forgetting factor for RLS}

The PSD plots show that much of the energy for the signals in within the frequency range of $0 \, Hz$ to $0.05 \, Hz$. Thus, the forgetting factor for the RLS algorithm is chosen such that the equivalent time-constant is greater than $1/0.05 = 20 \, s$. We have,

\begin{align*}
        T_f &= \frac{h}{1 - \lambda}\\
        T_f = 20 \quad &\text{ and } \quad h = 1\\
        \implies \lambda &= 1 - \frac{h}{T_f} = 1 - \frac{1}{20} = 0.95
\end{align*}
